\subsection*{Introduction }

This app launches the modules necessary to calibrate event-\/driven cameras. The event stream is sent in form of B\+L\+OB events (see \href{http://robotology.github.io/event-driven/doxygen/doc/html/group__vFramer.html}{\tt v\+Framer} for further details on the event types.) to the \href{http://wiki.icub.org/iCub/main/dox/html/group__icub__stereoCalib.html}{\tt stereo\+Calib} module. In the app we already provide a set of default parameters to the stereo\+Calib module defining the calibration board type as the asymmetric circle grid (shown in image below) which have proven to be optimal for event cameras calibration.



In the following image an overview of the opened ports and how they are connected.



\subsection*{Dependencies }

No special dependencies are required, all the required modules will be executed by the application.

\subsection*{How to run the application }


\begin{DoxyItemize}
\item On a console, run yarpserver (if not already running).
\item On another console run yarpmanager
\item Inside the Application folder in the yarpmanager gui, you should see an entry called v\+Calib. Double click and open it. Check that the parameters of the stereo\+Calib module are coherent with the board type you are going to use for calibration.
\item Run the application and connect the ports with the buttons in the yarpmanager G\+UI. You will now see the yarpview windows displaying the events.
\item On a terminal type \begin{DoxyVerb}  yarp rpc /stereoCalib/cmd
\end{DoxyVerb}

\end{DoxyItemize}

You can now send commands to the stereo\+Calib module.
\begin{DoxyItemize}
\item Type {\ttfamily start} in the command prompt. You should get the following output\+: \begin{DoxyVerb}    >>start
    Response: "Starting Calibration..."
\end{DoxyVerb}

\end{DoxyItemize}

Now the image acquisition has started.
\begin{DoxyItemize}
\item Move the calibration board in front of the cameras until the module recognises the board as many times as specified within the stereo\+Calib parameters (30 by default).
\item Once the calibration is over the results are written in the {\ttfamily \$\+I\+C\+U\+B\+\_\+\+D\+IR/contexts/camera\+Calibration/output\+Calib.ini}.
\item If you are satisfied with the results, you can copy them into the camera configuration file. 
\end{DoxyItemize}