The most important thing is to make sure all computers that are using {\ttfamily Y\+A\+RP} together are on the same subnet, and that the {\ttfamily yarpserver} is running on the same sub-\/net. Most problems with connection are because one of these is not true. Here\textquotesingle{}s the standard way to do if for {\ttfamily event-\/driven}.

\paragraph*{Run the {\ttfamily yarpserver} on the Z\+CB}

On your S\+SH connection to the Z\+CB\+: 
\begin{DoxyCode}
yarp namespace /<your name>
yarp conf <ip address> 10000
yarpserver
\end{DoxyCode}
 where{\ttfamily $<$ip address$>$} is the IP address assigned to the Z\+CB eth0 connection. If you want to run the {\ttfamily yarpserver} and the {\ttfamily zynq\+Grabber} on the same S\+SH connection, you can run the {\ttfamily yarpserver} in the background\+: 
\begin{DoxyCode}
yarpserver &
\end{DoxyCode}
 you can use 
\begin{DoxyCode}
fg
\end{DoxyCode}
 to bring {\ttfamily yarpserver} back to the foreground.

\paragraph*{Connect your laptop to the {\ttfamily yarpserver}}

On a terminal on your own laptop\+: 
\begin{DoxyCode}
yarp conf <ip address> 10000
\end{DoxyCode}
 {\bfseries Note\+:} this is the ip address of the Z\+C\+B! not your laptop! \+:warning\+: 
\begin{DoxyCode}
yarp detect
\end{DoxyCode}
 should find the {\ttfamily yarpserver} you have running on the zcb. 